%% BioMed_Central_Tex_Template_v1.05
%%                                      %
%  bmc_article.tex            ver: 1.05 %
%                                       %

%%%%%%%%%%%%%%%%%%%%%%%%%%%%%%%%%%%%%%%%%%%%%%%%%%%%%%%%%%%%%%%%%%%%%
%%                                                                 %%	
%% For instructions on how to fill out this Tex template           %%
%% document please refer to Readme.pdf and the instructions for    %%
%% authors page on the biomed central website                      %%
%% http://www.biomedcentral.com/info/authors/                      %%
%%                                                                 %%
%% Please do not use \input{...} to include other tex files.       %%
%% Submit your LaTeX manuscript as one .tex document.              %%
%%                                                                 %%
%% All additional figures and files should be attached             %%
%% separately and not embedded in the \TeX\ document itself.       %%
%%                                                                 %%
%% BioMed Central currently use the MikTex distribution of         %%
%% TeX for Windows) of TeX and LaTeX.  This is available from      %%
%% http://www.miktex.org                                           %%
%%                                                                 %%
%%%%%%%%%%%%%%%%%%%%%%%%%%%%%%%%%%%%%%%%%%%%%%%%%%%%%%%%%%%%%%%%%%%%%


\NeedsTeXFormat{LaTeX2e}[1995/12/01]
\documentclass[10pt]{bmc_article}    

% Load packages
\usepackage{cite} % Make references as [1-4], not [1,2,3,4]
\usepackage{url}  % Formatting web addresses  
\usepackage{ifthen}  % Conditional 
\usepackage{multicol}   %Columns
\usepackage[utf8]{inputenc} %unicode support
%\usepackage[applemac]{inputenc} %applemac support if unicode package fails
%\usepackage[latin1]{inputenc} %UNIX support if unicode package fails
\urlstyle{rm} 
 
%%%%%%%%%%%%%%%%%%%%%%%%%%%%%%%%%%%%%%%%%%%%%%%%%	
%%                                             %%
%%  If you wish to display your graphics for   %%
%%  your own use using includegraphic or       %%
%%  includegraphics, then comment out the      %%
%%  following two lines of code.               %%   
%%  NB: These line *must* be included when     %%
%%  submitting to BMC.                         %% 
%%  All figure files must be submitted as      %%
%%  separate graphics through the BMC          %%
%%  submission process, not included in the    %% 
%%  submitted article.                         %% 
%%                                             %%
%%%%%%%%%%%%%%%%%%%%%%%%%%%%%%%%%%%%%%%%%%%%%%%%%               

%\def\includegraphic{}
%\def\includegraphics{}
\usepackage[]{graphicx}

\setlength{\topmargin}{0.0cm}
\setlength{\textheight}{21.5cm}
\setlength{\oddsidemargin}{0cm} 
\setlength{\textwidth}{16.5cm}
\setlength{\columnsep}{0.6cm}

\newboolean{publ}

%%%%%%%%%%%%%%%%%%%%%%%%%%%%%%%%%%%%%%%%%%%%%%%%%%
%%                                              %%
%% You may change the following style settings  %%
%% Should you wish to format your article       %%
%% in a publication style for printing out and  %%
%% sharing with colleagues, but ensure that     %%
%% before submitting to BMC that the style is   %%
%% returned to the Review style setting.        %%
%%                                              %%
%%%%%%%%%%%%%%%%%%%%%%%%%%%%%%%%%%%%%%%%%%%%%%%%%%
 

%Review style settings
\newenvironment{bmcformat}{\begin{raggedright}\baselineskip20pt\sloppy\setboolean{publ}{false}}{\end{raggedright}\baselineskip20pt\sloppy}

%Publication style settings
%\newenvironment{bmcformat}{\fussy\setboolean{publ}{true}}{\fussy}

% Begin ...
\begin{document}
\begin{bmcformat}

\title{Predicting the transmission vector of Influenza using linear
  peptide motifs}
 
%%%%%%%%%%%%%%%%%%%%%%%%%%%%%%%%%%%%%%%%%%%%%%
%%                                          %%
%% Enter the authors here                   %%
%%                                          %%
%% Ensure \and is entered between all but   %%
%% the last two authors. This will be       %%
%% replaced by a comma in the final article %%
%%                                          %%
%% Ensure there are no trailing spaces at   %% 
%% the ends of the lines                    %%     	
%%                                          %%
%%%%%%%%%%%%%%%%%%%%%%%%%%%%%%%%%%%%%%%%%%%%%%

\author{Perry Evans\correspondingauthor$^{1,2}$%
       \email{samesense@gmail.com\correspondingauthor - charles@londonzoo.co.uk}%
      \and
         Will Dampier\correspondingauthor$^2$%
         \email{Jane E Doe\correspondingauthor - jane.e.doe@cambridge.co.uk}
          \and
         Yichuan Liu\correspondingauthor$^2$%
         \email{Jane E Doe\correspondingauthor - jane.e.doe@cambridge.co.uk}
          \and
         Lyle Ungar\correspondingauthor$^2$%
         \email{Jane E Doe\correspondingauthor - jane.e.doe@cambridge.co.uk}
       and 
         Aydin Tozeren$^3$%
         \email{John RS Smith - john.RS.Smith@cambridge.co.uk}%
      }
      
%%%%%%%%%%%%%%%%%%%%%%%%%%%%%%%%%%%%%%%%%%%%%%
%%                                          %%
%% Enter the authors' addresses here        %%
%%                                          %%
%%%%%%%%%%%%%%%%%%%%%%%%%%%%%%%%%%%%%%%%%%%%%%

\address{%
    \iid(1)Life Sciences Department, Kings College London, Cornwall House,%
        Waterloo Road, London, UK\\
    \iid(2)Department of Zoology, Cambridge, Waterloo Road, London, UK\\
    \iid(3)Marine Ecology Department, Institute of Marine Sciences Kiel, %
        D\"{u}sternbrooker Weg 20, 24105 Kiel, Germany
}%

\maketitle

%%%%%%%%%%%%%%%%%%%%%%%%%%%%%%%%%%%%%%%%%%%%%%
%%                                          %%
%% The Abstract begins here                 %%
%%                                          %%
%% The Section headings here are those for  %%
%% a Research article submitted to a        %%
%% BMC-Series journal.                      %%  
%%                                          %%
%% If your article is not of this type,     %%
%% then refer to the Instructions for       %%
%% authors on http://www.biomedcentral.com  %%
%% and change the section headings          %%
%% accordingly.                             %%   
%%                                          %%
%%%%%%%%%%%%%%%%%%%%%%%%%%%%%%%%%%%%%%%%%%%%%%

\begin{abstract}
        % Do not use inserted blank lines (ie \\) until main body of text.
        \paragraph*{Background:} Virus proteins harbor host peptide motifs that enable them to carry out necessary steps in their life-cycles by interacting with host proteins. The influenza virus infects many hosts, and its nucleotide sequence has been shown to evolve differently based on the host it infects. Here we examine differential peptide motif usage for inflenza viruses taken from different hosts, and check to see if virus differences correspond with those of its hosts.  
      
        \paragraph*{Results:} We annotate virus and host protein sequences with 153 confirmed peptide motifs, and cluster sequence hits based on edit distance.  For clusters common to all hosts and virus groups, we find distributions of cluster usage for each host and virus group. We take the Jensen-Shannon divergence between distributions for hosts and influenza groups, and find that each influenza group matches best with its correct host.

        \paragraph*{Conclusions:} Peptide motif usage differs between hosts and between viruses derived from different hosts. Virus motif usage is most similar to the host from which it was taken.

        \paragraph*{Availability:} The source code and paper, including all revisions, are available at http://github.com/JudoWill/flELM.
        
\end{abstract}

\ifthenelse{\boolean{publ}}{\begin{multicols}{2}}{}

%%%%%%%%%%%%%%%%%%%%%%%%%%%%%%%%%%%%%%%%%%%%%%
%%                                          %%
%% The Main Body begins here                %%
%%                                          %%
%% The Section headings here are those for  %%
%% a Research article submitted to a        %%
%% BMC-Series journal.                      %%  
%%                                          %%
%% If your article is not of this type,     %%
%% then refer to the instructions for       %%
%% authors on:                              %%
%% http://www.biomedcentral.com/info/authors%%
%% and change the section headings          %%
%% accordingly.                             %% 
%%                                          %%
%% See the Results and Discussion section   %%
%% for details on how to create sub-sections%%
%%                                          %%
%% use \cite{...} to cite references        %%
%%  \cite{koon} and                         %%
%%  \cite{oreg,khar,zvai,xjon,schn,pond}    %%
%%  \nocite{smith,marg,hunn,advi,koha,mouse}%%
%%                                          %%
%%%%%%%%%%%%%%%%%%%%%%%%%%%%%%%%%%%%%%%%%%%%%%

%%%%%%%%%%%%%%%%
%% Background %%
%%
\section*{Background}
Influenza A virus has the ability to infect a number of host
organisms, including human, chicken, swine, and horse. Nucleotide
sequences of the 2009 H1N1 pandemic influenza have been shown to have
a subsitution bias that depends on the host organism in which it
resides \cite{solovyov2010host}. In this study we examine the
possibility of a peptide motif usage bias that correlates with host
organism.

Virus proteins use many host peptide motifs to interact with host
proteins to accomplish necessary steps in the viral life-cycle
\cite{kadaveru13viral}. Host motifs on HIV proteins have been used to
predict patient response to therapy and interations with host proteins
\cite{chan2009decoding}.

Peptide sequence usage has been used to profile prokaryotic proteomes,
and the distances between profiles accuratly recovers the known
prokaryotic phylogeny \cite{jun2010whole}.

Here we demonstrate that the usage of peptide motifs differs between
influenza groups and host organisms. We show that an influenza group's
peptide motif profile if closest that of its host.
 
%%%%%%%%%%%%%%%%%%%%%%%%%%%%
%% Results and Discussion %%
%%
\section*{Results and Discussion}
  \subsection*{ELM profiles differ between hosts}

  \subsection*{ELM profiles differ between influenza groups}

  \subsection*{ELM profiles recover host phylogeny}

  \subsection*{ELM sequence cluster profiles recover host phylogeny}

  \subsection*{ELM sequence cluster profiles assign correct host-influenza pairings}


    

%%%%%%%%%%%%%%%%%%%%%%
\section*{Conclusions}
Here we investigated influenza group peptide biases, and checked to
see if they corresponded with those of their hosts.
  
%%%%%%%%%%%%%%%%%%
\section*{Methods}

\subsection*{Sequence Data}
All protein sequences used in the study were downloaded from
NCBI. Host organisms in NCBI have different degrees of protein
coverage. To standardize host and influenza sequence comparisons, we
used ortholgs common to all host species. Roundup
\cite{deluca2006roundup} was used with default settings to determine
common ortholgos for human, chimpanzee, mouse, rat, cow, dog, chicken
and zebra finch. Roundup did not provide orthologs for horse or pig,
so these were determined using the reciprocal smallest distance
\cite{wall2003detecting}.

NCBI influenza protein sequences are tagged with the host organism
they were taken from. For this study, we used influenza A sequences
taken from human, equine, swine, chicken, or duck. We further limited
influenza sequences by using only those that corresponded to proteins
hemagglutinin, neuraminidase, nucleocapsid protein, matrix protein 1,
matrix protein 2, nonstructural protein 1, nonstructural protein 2,
polymerase PB1, polymerase PA, polymerase PB2, or PB1-F2 protein. We
organized influenza sequences into groups based on the host organism
with which there were tagged.

All protein sequences were matched against regular expressions for 153
eukaryotic linear motifs cataloged in the ELM resource
\cite{gould2010elm}. Rapid matching was facilitated using cloud
computing provided by PiCloud.

\subsection*{Phylogenies}
Phylogenies were constructed using hierarchical clustering with
average linkage. Distances between groups were found using
Jensen-Shannon divergence.

\subsection*{Clustering}
The host orgamisms had hits for all ELM patterns, but the influenza
sequences only had hits for XXX. The sequences matched by the ELM
patterns were not identical for host and influenza. Most of the
sequence matches on influenza were not present on host orgamisms. For
this reason, we decided to cluster ELM sequence matches based on the
Levenshtein distance \cite{levenshteiti1966binary}, which is the
minimun number of edits required to transform one sequence into
another.

For each ELM whose pattern matched an influenza sequence, we clustered
all host and influenza pattern matches in two steps.  In the first
step, we made clusters for each inflenza pattern match, and added to
these clusters any host organism match that was at most two edits away
from the influenza string. This led to many overlapping clusters. In
the second step, we used k-medoids to cluster the overlapping clusters
using the common sequences between two clusters as a distance
metric. 

It was possible that the resulting meta-clusters shared host
ELM sequences. In the second step of clustering, we chose k to be as
large as possible, but constrained it so that the number of
overlapping sequences between to meta-clusters was small than ten
percent of the total sequences beings clustered. The sequences that
were common to two or more clusters were removed from the
meta-clusters and reassigned based on averge linkage.

Our clustering method uses three arbitrary cutoffs. In the first
clustering step, we assign host ELM sequences to clusters based on an
edit distance less than two. In the second clustering step, we choose
a ten percent threshold to limit the sequences shared by two
meta-clusters.

\subsection*{Comparing ELM sequence cluster distributions}
With our ELM sequence clusters established, we found distributions of
cluster memebership for all host organisms and influenza groups.  We
standardized these ELM sequence cluster profiles by only considering
ELM sequence clusters with members that appears in all hosts and all
influenza groups. We used Jensen-Shannon divergence
\cite{lin1991divergence} to compare ELM sequence profiles taken from
two host organisms or a host organism and an influenza group.
  
%%%%%%%%%%%%%%%%%%%%%%%%%%%%%%%%
\section*{Authors contributions}
    All authors contributed conceptually. For specific contributions,
    see this page: http://github.com/JudoWill/flELM/graphs/impact.

%%%%%%%%%%%%%%%%%%%%%%%%%%%
\section*{Acknowledgements}
  \ifthenelse{\boolean{publ}}{\small}{}
  This study was supported by the National Institutes of Health
  (NIH) grant number \# 232240 and by the National Science Foundation
  grant \# 235327. Additional support came from a Drexel University
  Calhoun Fellowship (WD) and from NIH training grant T32 HG000046
  (PE).
 
%%%%%%%%%%%%%%%%%%%%%%%%%%%%%%%%%%%%%%%%%%%%%%%%%%%%%%%%%%%%%
%%                  The Bibliography                       %%
%%                                                         %%              
%%  Bmc_article.bst  will be used to                       %%
%%  create a .BBL file for submission, which includes      %%
%%  XML structured for BMC.                                %%
%%                                                         %%
%%                                                         %%
%%  Note that the displayed Bibliography will not          %% 
%%  necessarily be rendered by Latex exactly as specified  %%
%%  in the online Instructions for Authors.                %% 
%%                                                         %%
%%%%%%%%%%%%%%%%%%%%%%%%%%%%%%%%%%%%%%%%%%%%%%%%%%%%%%%%%%%%%

%% {\ifthenelse{\boolean{publ}}{\footnotesize}{\small}
%%  \bibliographystyle{bmc_article}  % Style BST file
%%   \bibliography{refs} }     % Bibliography file (usually '*.bib' ) 
\bibliographystyle{bmc_article}  % Style BST file
  \bibliography{refs} 
%%%%%%%%%%%

\ifthenelse{\boolean{publ}}{\end{multicols}}{}

%%%%%%%%%%%%%%%%%%%%%%%%%%%%%%%%%%%
%%                               %%
%% Figures                       %%
%%                               %%
%% NB: this is for captions and  %%
%% Titles. All graphics must be  %%
%% submitted separately and NOT  %%
%% included in the Tex document  %%
%%                               %%
%%%%%%%%%%%%%%%%%%%%%%%%%%%%%%%%%%%

%%
%% Do not use \listoffigures as most will included as separate files

\section*{Figures}
  \subsection*{Figure 1 - Host ELM sequence usage}
      A short description of the figure content
      should go here.

      \begin{figure}
      \begin{center}
      \includegraphics{figs/fig1}
      \end{center}
      \end{figure}

  \subsection*{Figure 2 - Influenza group ELM sequence usage}
      Figure legend text.

%%%%%%%%%%%%%%%%%%%%%%%%%%%%%%%%%%%
%%                               %%
%% Tables                        %%
%%                               %%
%%%%%%%%%%%%%%%%%%%%%%%%%%%%%%%%%%%

%% Use of \listoftables is discouraged.
%%
\section*{Tables}
  \subsection*{Table 1 - Jensen-Shannon divergece between influenza groups and hosts}
    Here is an example of a \emph{small} table in \LaTeX\ using  
    \verb|\tabular{...}|. This is where the description of the table 
    should go. \par \mbox{}
    \par
    \mbox{
      \begin{tabular}{|c|c|c|}
        \hline \multicolumn{3}{|c|}{My Table}\\ \hline
        A1 & B2  & C3 \\ \hline
        A2 & ... & .. \\ \hline
        A3 & ..  & .  \\ \hline
      \end{tabular}
      }

%%%%%%%%%%%%%%%%%%%%%%%%%%%%%%%%%%%
%%                               %%
%% Additional Files              %%
%%                               %%
%%%%%%%%%%%%%%%%%%%%%%%%%%%%%%%%%%%

\section*{Additional Files}
  \subsection*{Additional file 1 --- Sample additional file title}
    Additional file descriptions text (including details of how to
    view the file, if it is in a non-standard format or the file extension).  This might
    refer to a multi-page table or a figure.

\end{bmcformat}
\end{document}







